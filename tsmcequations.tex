\documentclass{article}
\usepackage{amsmath}
\usepackage[margin=1in]{geometry}

\DeclareMathOperator{\Pois}{Pois}
\DeclareMathOperator{\Prob}{P}
\DeclareMathOperator{\Exp}{Exp}
\DeclareMathOperator{\Bernoulli}{Bernoulli}
\DeclareMathOperator{\GammaDist}{Gamma}
\DeclareMathOperator{\NegBin}{NegBin}
\DeclareMathOperator{\Erlang3}{Erlang-3}
\DeclareMathOperator{\E}{E}
\DeclareMathOperator{\Var}{Var}
\DeclareMathOperator{\Cov}{Cov}
\DeclareMathOperator*{\argmax}{\arg\!\max}
\def\bs{\boldsymbol}
\def\ibd{\textrm{IBD}}

\def\H{\mathcal{H}}
\def\R{\mathcal{R}}
\def\X{\bs{X}}
\def\m{\mathrm{(m)}}
\def\p{\mathrm{(p)}}
\def\a{\mathrm{(a)}}
\def\b{\mathrm{(b)}}
\def\P{\mathcal{P}}
\def\U{\mathcal{U}}

\begin{document}

\section{General equations}

Define $N(t)$ as the population size at time $t$, with $t$ in units of $2N(0)$.
Let $\lambda(t)$ be the relative population size such that $N(t) =
N(0)\lambda(t)$. Define $\Omega(u,v)$ as the cumulative coalescent rate between
times $u$ and $v$:

\begin{equation}
    \Omega(u,v) = \int_u^v \frac{dt}{\lambda(t)}.
\end{equation}

The state of the TSMC at each point along the genome is described by the vector
$\boldsymbol{s} = (s_3,s_2)$, where $s_3$ is the first coalescence time and
$s_2$ is the second coalescence time amongst the three lineages in a triploid
genome. Let $q(\boldsymbol{t}|\boldsymbol{s})$ be the transition kernel at
recombination sites along the genome. Then

\begin{align}
    \begin{split}
    q(\bs{t}|\bs{s}) &= \\ 
    \intertext{For $t_3=s_3;t_2>s_2$:}
    &\int_0^{s_3}\frac{du}{2s_2+s_3}e^{-3\Omega(u,s_3)}e^{-2\Omega(s_3,s_2)}\frac{1}{\lambda(t_2)}e^{-\Omega(s_2,t_2)}+
    2\int_{s_3}^{s_2}\frac{du}{2s_2+s_3}e^{-\Omega(u,s_2)}\frac{1}{\lambda(t_2)}e^{-\Omega(s_2,t_2)}\\
    \intertext{For $t_3=s_3; t_2<s_2$:}
    &\int_0^{s_3}\frac{du}{2s_2+s_3}e^{-3\Omega(u,s_3)}\frac{1}{\lambda(t_2)}e^{-2\Omega(s_3,t_2)}+
    2\int_{s_3}^{t_2}\frac{du}{2s_2+s_3}\frac{1}{\lambda(t_2)}e^{-2\Omega(u,t_2)}\\
    \intertext{For $t_3<s_3; t_2=s_3$:}
    &\int_0^{t_3}\frac{du}{2s_2+s_3}e^{-3\Omega(u,t_3)}\frac{2}{\lambda(t_3)}\\
    \intertext{For $t_3<s_3; t_2=s_2$:}
    &2\int_0^{t_3}\frac{du}{2s_2+s_3}\frac{2}{\lambda(t_3)}e^{-3\Omega(u,t_3)}\\
    \intertext{For $t_3>s_3; t_2=s_2$:}
    &2\int_0^{s_3}\frac{du}{2s_2+s_3}e^{-3\Omega(u,s_3)}\frac{2}{\lambda(t_3)}e^{-2\Omega(s_3,t_3)}\\
    \intertext{For $t_3=s_2; t_2>s_2$:}
    &2\int_0^{s_3}\frac{du}{2s_2+s_3}e^{-3\Omega(u,s_3)}e^{-2\Omega(s_3,s_2)}\frac{1}{\lambda(t_2)}
    \intertext{For $t_3=s_2; t_2=s_2$:}
    &3\int_0^{s_3}\frac{du}{2s_2+s_3}\frac{1}{3}\left[1-e^{-3\Omega(u,s_3)}\right]+\int_0^{s_3}\frac{du}{2s_2+s_3}e^{-3\Omega(u,s_3)}\frac{1}{2}\left[1-e^{-2\Omega(s_3,s_2)}\right]+2\int_{s_3}^{s_2}\frac{du}{2s_2+s_3}\frac{1}{2}\left[1-e^{-2\Omega(u,s_2)}\right].
\end{split}
\label{eq:qts}
\end{align}

\section{Piecewise constant transition probabilities}

Suppose that the population changes size at times $(T_0, T_1, \dots, T_n)$ and
that the size between $T_i$ and $T_{i+1}$ is a constant $2N\lambda_i$. Define
$T_{n+1} = \infty$ and $\Delta_i = T_{i+1}-T_i$. Let $\alpha(t)$ be the index
of the time interval to which $t$ belongs, \emph{i.e.}, $\alpha(t) = \max_i
\{i:T_i \le t\}$.

Then the cumulative coalescent rate between $u$ and $v$ can be written

\begin{equation}
    \Omega(u,v) =
    \begin{cases}
        \frac{v-u}{\lambda_{\alpha(u)}} & \alpha(u) = \alpha(v)\\
        \left(T_{\alpha(u)+1}-u\right)\frac{1}{\lambda_{\alpha(u)}} +
          \sum_{i=\alpha(u)+1}^{\alpha(v)-1}\frac{\Delta_i}{\lambda_i} +
          \left(v-T_{\alpha(v)}\right)\frac{1}{\lambda_{\alpha(v)}}
          & \alpha(u) < \alpha(v).
    \end{cases}
\end{equation}


There are several integrals of the form $\int_x^y e^{-k\Omega(u,y)}du$ in
Equation \eqref{eq:qts}. This integral can be written

\begin{align}
    \begin{split}
        \int_x^y e^{-k(y-u)}du &= \int_x^{T_{\alpha(x)+1}} e^{-k\Omega(u,y)}du +
        \sum_{i=\alpha(x)+1}^{\alpha(y)-1}\int_{T_i}^{T_i+1}e^{-k\Omega(u,y)}du +
    \int_{T_\alpha(y)}^y e^{-k\Omega(u,y)}du \\
    &= \int_x^{T_{\alpha(x)+1}} e^{-k\Omega(u,y)}du
    %TODO here
    \end{split}
\end{align}

\end{document}
