\documentclass{article}
\usepackage{amsmath}
\usepackage[margin=1in]{geometry}

\DeclareMathOperator{\Pois}{Pois}
\DeclareMathOperator{\Prob}{P}
\DeclareMathOperator{\Exp}{Exp}
\DeclareMathOperator{\Bernoulli}{Bernoulli}
\DeclareMathOperator{\GammaDist}{Gamma}
\DeclareMathOperator{\NegBin}{NegBin}
\DeclareMathOperator{\Erlang3}{Erlang-3}
\DeclareMathOperator{\E}{E}
\DeclareMathOperator{\Var}{Var}
\DeclareMathOperator{\Cov}{Cov}
\DeclareMathOperator*{\argmax}{\arg\!\max}
\def\bs{\boldsymbol}
\def\ibd{\textrm{IBD}}

\def\H{\mathcal{H}}
\def\R{\mathcal{R}}
\def\X{\bs{X}}
\def\m{\mathrm{(m)}}
\def\p{\mathrm{(p)}}
\def\a{\mathrm{(a)}}
\def\b{\mathrm{(b)}}
\def\P{\mathcal{P}}
\def\U{\mathcal{U}}

\begin{document}

\section{General equations}

Define $N(t)$ as the population size at time $t$, with $t$ in units of $2N(0)$.
Let $\lambda(t)$ be the relative population size such that $N(t) =
N(0)\lambda(t)$. Define $\Omega(u,v)$ as the cumulative coalescent rate between
times $u$ and $v$:

\begin{equation}
    \Omega(u,v) = \int_u^v \frac{dt}{\lambda(t)}.
\end{equation}

The state of the TSMC at each point along the genome is described by the vector
$\boldsymbol{s} = (s_3,s_2)$, where $s_3$ is the first coalescence time and
$s_2$ is the second coalescence time amongst the three lineages in a triploid
genome. The equilibrium joint distribution of $(t_3,t_2)$ is

\begin{equation}
    \pi(t_3,t_2) = \frac{1}{\lambda(t_3)\lambda(t_2)}e^{-3\Omega(0,t_3)}e^{-2\Omega(t_3,t_2)}.
    \label{eq:equilibriumjointdistn}
\end{equation}

Let $q(\boldsymbol{t}|\boldsymbol{s})$ be the transition kernel at
recombination sites along the genome. Then

\begin{align}
    \begin{split}
    q(\bs{t}|\bs{s}) &= \\ 
    \intertext{For $t_3=s_3;t_2>s_2$:}
    &\int_0^{s_3}\frac{du}{2s_2+s_3}e^{-3\Omega(u,s_3)}e^{-2\Omega(s_3,s_2)}\frac{1}{\lambda(t_2)}e^{-\Omega(s_2,t_2)}+
    2\int_{s_3}^{s_2}\frac{du}{2s_2+s_3}e^{-2\Omega(u,s_2)}\frac{1}{\lambda(t_2)}e^{-\Omega(s_2,t_2)}\\
    \intertext{For $t_3=s_3; t_2<s_2$:}
    &\int_0^{s_3}\frac{du}{2s_2+s_3}e^{-3\Omega(u,s_3)}\frac{1}{\lambda(t_2)}e^{-2\Omega(s_3,t_2)}+
    2\int_{s_3}^{t_2}\frac{du}{2s_2+s_3}\frac{1}{\lambda(t_2)}e^{-2\Omega(u,t_2)}\\
    \intertext{For $t_3<s_3; t_2=s_3$:}
    &\int_0^{t_3}\frac{du}{2s_2+s_3}e^{-3\Omega(u,t_3)}\frac{2}{\lambda(t_3)}\\
    \intertext{For $t_3<s_3; t_2=s_2$:}
    &2\int_0^{t_3}\frac{du}{2s_2+s_3}\frac{2}{\lambda(t_3)}e^{-3\Omega(u,t_3)}\\
    \intertext{For $t_3>s_3; t_2=s_2$:}
    &2\int_0^{s_3}\frac{du}{2s_2+s_3}e^{-3\Omega(u,s_3)}\frac{2}{\lambda(t_3)}e^{-2\Omega(s_3,t_3)}\\
    \intertext{For $t_3=s_2; t_2>s_2$:}
    &2\int_0^{s_3}\frac{du}{2s_2+s_3}e^{-3\Omega(u,s_3)}e^{-2\Omega(s_3,s_2)}\frac{1}{\lambda(t_2)}e^{-\Omega(s_2,t_2)}
    \intertext{For $t_3=s_3; t_2=s_2$:}
    &3\int_0^{s_3}\frac{du}{2s_2+s_3}\frac{1}{3}\left[1-e^{-3\Omega(u,s_3)}\right]+\int_0^{s_3}\frac{du}{2s_2+s_3}e^{-3\Omega(u,s_3)}\frac{1}{2}\left[1-e^{-2\Omega(s_3,s_2)}\right]+2\int_{s_3}^{s_2}\frac{du}{2s_2+s_3}\frac{1}{2}\left[1-e^{-2\Omega(u,s_2)}\right].
\end{split}
\label{eq:qts}
\end{align}

Each part is implicitly multiplied by a delta function to limit the density to
points where the parameters are assumed to be equal to each other. For example,
the first part of $q(\bs{t}|\bs{s})$ is implicitly multiplied by
$\delta(t_3-s_3)$, and the last part is multiplied by
$\delta(t_3-s_3)\delta(t_2-s_2)$.

\section{Piecewise constant transition probabilities}

Suppose that the population changes size at times $(T_0, T_1, \dots, T_n)$ and
that the size between $T_i$ and $T_{i+1}$ is a constant $2N\lambda_i$. Define
$T_{n+1} = \infty$ and $\Delta_i = T_{i+1}-T_i$. Let $\alpha(t)$ be the index
of the time interval to which $t$ belongs, \emph{i.e.}, $\alpha(t) = \max_i
\{i:T_i \le t\}$.

Then the cumulative coalescent rate between $u$ and $v$ can be written

\begin{equation}
    \Omega(u,v) =
    \begin{cases}
        \frac{v-u}{\lambda_{\alpha(u)}} & \alpha(u) = \alpha(v)\\
        \left(T_{\alpha(u)+1}-u\right)\frac{1}{\lambda_{\alpha(u)}} +
          \sum_{i=\alpha(u)+1}^{\alpha(v)-1}\frac{\Delta_i}{\lambda_i} +
          \left(v-T_{\alpha(v)}\right)\frac{1}{\lambda_{\alpha(v)}}
          & \alpha(u) < \alpha(v).
    \end{cases}
\end{equation}
The equilibrium joint density of $(t_3,t_2)$ is now approximately

\begin{equation}
    \pi(t_3,t_2) = \frac{1}{\lambda_{\alpha(t_3)}\lambda_{\alpha(t_2)}}e^{-3\Omega(0,t_3)}e^{-2\Omega(t_3,t_2)}.
    \label{eq:marginal}
\end{equation}



There are several integrals of the form $\int_x^y e^{-k\Omega(u,y)}du$ in
Equation \eqref{eq:qts}. This integral can be written

\begin{align}
    \begin{split}
        \int_x^y e^{-k\Omega(u,y)}du &= 
        \int_x^{T_{\alpha(x)+1}} e^{-k\Omega(u,y)}du +
        \sum_{i=\alpha(x)+1}^{\alpha(y)-1}\int_{T_i}^{T_i+1}e^{-k\Omega(u,y)}du +
    \int_{T_\alpha(y)}^y e^{-k\Omega(u,y)}du \\
    &= \int_x^{T_{\alpha(x)+1}} \exp\left(-k\left[\left(T_{\alpha(u)+1}-u\right)\frac{1}{\lambda_{\alpha(u)}} + 
    \sum_{j=\alpha(u)+1}^{\alpha(y)-1}\frac{\Delta_j}{\lambda_j} + (y-T_{\alpha(y)})\frac{1}{\lambda_{\alpha(y)}}
    \right]\right)du + \\
    &\qquad \qquad \sum_{i=\alpha(x)+1}^{\alpha(y)-1}\int_{T_i}^{T_{i+1}}
    \exp\left(-k\left[\left(T_{\alpha(u)+1}-u\right)\frac{1}{\lambda_{\alpha(u)}} + 
    \sum_{j=\alpha(u)+1}^{\alpha(y)-1}\frac{\Delta_j}{\lambda_j} + (y-T_{\alpha(y)})\frac{1}{\lambda_{\alpha(y)}}
    \right]\right)du + \\
    &\qquad \qquad \int_{T_{\alpha(y)}}^y \exp\left(-k(y-u)\frac{1}{\lambda_{\alpha(u)}}\right)du \\
    % subst. appropriate \alpha(x)'s etc. for \alpha(u)
    &= \int_x^{T_{\alpha(x)+1}} \exp\left(-k\left[\left(T_{\alpha(x)+1}-u\right)\frac{1}{\lambda_{\alpha(x)}} + 
    \sum_{j=\alpha(x)+1}^{\alpha(y)-1}\frac{\Delta_j}{\lambda_j} + (y-T_{\alpha(y)})\frac{1}{\lambda_{\alpha(y)}}
    \right]\right)du + \\
    &\qquad \qquad \sum_{i=\alpha(x)+1}^{\alpha(y)-1}\int_{T_i}^{T_{i+1}}
    \exp\left(-k\left[\left(T_{i+1}-u\right)\frac{1}{\lambda_{i}} + 
    \sum_{j=i+1}^{\alpha(y)-1}\frac{\Delta_j}{\lambda_j} + (y-T_{\alpha(y)})\frac{1}{\lambda_{\alpha(y)}}
    \right]\right)du + \\
    &\qquad \qquad \int_{T_{\alpha(y)}}^y \exp\left(-k(y-u)\frac{1}{\lambda_{\alpha(y)}}\right)du \\
    % move constants out of exponent
    &= \exp\left(-k\left[\sum_{j=\alpha(x)+1}^{\alpha(y)-1}\frac{\Delta_j}{\lambda_j} +
        (y-T_{\alpha(y)})\frac{1}{\lambda_{\alpha(y)}}\right]\right) 
        \int_x^{T_{\alpha(x)+1}} \exp\left(-k\left(T_{\alpha(x)+1}-u\right)\frac{1}{\lambda_{\alpha(x)}}\right)du + \\
    &\qquad\qquad \sum_{i=\alpha(x)+1}^{\alpha(y)-1}
        \exp\left(-k\left[\sum_{j=i+1}^{\alpha(y)-1}\frac{\Delta_j}{\lambda_j} + (y-T_{\alpha(y)})\frac{1}{\lambda_{\alpha(y)}}
        \right]\right)
    \int_{T_i}^{T_{i+1}}\exp\left(-k\left(T_{i+1}-u\right)\frac{1}{\lambda_{i}}\right)du + \\
    &\qquad \qquad \int_{T_{\alpha(y)}}^y \exp\left(-k(y-u)\frac{1}{\lambda_{\alpha(y)}}\right)du \\
    % solve integrals
    &= \exp\left(-k\left[\sum_{j=\alpha(x)+1}^{\alpha(y)-1}\frac{\Delta_j}{\lambda_j} +
        (y-T_{\alpha(y)})\frac{1}{\lambda_{\alpha(y)}}\right]\right) 
        \left[1-\exp\left(-\frac{k \left(T_{\alpha(x)+1}-x\right)}{\lambda_{\alpha(x)}}\right)\right]\frac{\lambda_{\alpha(x)}}{k}+\\
    &\qquad\qquad \sum_{i=\alpha(x)+1}^{\alpha(y)-1}
        \exp\left(-k\left[\sum_{j=i+1}^{\alpha(y)-1}\frac{\Delta_j}{\lambda_j} + (y-T_{\alpha(y)})\frac{1}{\lambda_{\alpha(y)}}
        \right]\right)
    \left[1-\exp\left(-\frac{k \Delta_i}{\lambda_i}\right)\right]\frac{\lambda_i}{k}+\\
    &\qquad \qquad 
    \left[1-\exp\left(-\frac{k \left(y-T_{\alpha(y)}\right)}{\lambda_{\alpha(y)}}\right)\right]\frac{\lambda_{\alpha(y)}}{k}\\
    % simplify back to \Omega function notation
    &= \exp\left(-k\Omega(T_{\alpha(x)+1}, y)\right)\left[1-\exp\left(-\frac{k \left(T_{\alpha(x)+1}-x\right)}{\lambda_{\alpha(x)}}\right)\right]\frac{\lambda_{\alpha(x)}}{k}+\\
    &\qquad\qquad \sum_{i=\alpha(x)+1}^{\alpha(y)-1} \exp\big(-k\Omega(T_{i+1},y)\big)
    \left[1-\exp\left(-\frac{k \Delta_i}{\lambda_i}\right)\right]\frac{\lambda_i}{k}+\\
    &\qquad \qquad 
    \left[1-\exp\left(-\frac{k \left(y-T_{\alpha(y)}\right)}{\lambda_{\alpha(y)}}\right)\right]\frac{\lambda_{\alpha(y)}}{k}\\
    % end eqn
    \end{split}
    \label{eq:piecewisenocoalintegral}
\end{align}

With this equation, we can calculate all of the transition probabilities in the
transition kernel \eqref{eq:qts}.

For $t_3=s_3;t_2>s_2$:
    % first summand
\begin{align*}
    &\frac{1}{2s_2+s_3}e^{-2\Omega(s_3,s_2)}\frac{1}{\lambda_{\alpha(t_2)}}e^{-\Omega(s_2,t_2)}
    \left\{\sum_{i=0}^{\alpha(s_3)-1}e^{-3\Omega(T_{i+1},s_3)}
        \left[1-e^{-\frac{3\Delta_i}{\lambda_i}}\right]\frac{\lambda_i}{3}+
    \left[1-e^{-\frac{3\left(s_3-T_{\alpha(s_3)}\right)}{\lambda_{\alpha(s_3)}}}\right]
        \frac{\lambda_{\alpha(s_3)}}{3}\right\}+\\
    % second summand
    &\qquad\frac{2}{2s_2+s_3}\frac{1}{\lambda_{\alpha(t_2)}}e^{-\Omega(s_2,t_2)}\times\\
    &\qquad\Bigg\{e^{-2\Omega(T_{\alpha(s_3)+1},s_2)}\left[1-e^{-\frac{2\left(T_{\alpha(s_3)+1}-s_3\right)}{\lambda_{\alpha(s_3)}}}\right]\frac{\lambda_{\alpha(s_3)}}{2}+\sum_{i=\alpha(s_3)+1}^{\alpha(s_2)-1}e^{-2\Omega\left(T_{i+1},s_2\right)}\left[1-e^{-\frac{2\Delta_i}{\lambda_i}}\right]\frac{\lambda_i}{2}\\
    &\qquad\qquad+\left[1-e^{-\frac{2\left(s_2-T_{\alpha(s_2)}\right)}{\lambda_{\alpha(s_2)}}}\right]\frac{\lambda_{\alpha(s_2)}}{2}
    \Bigg\}
    \\
    % (original)
    &\int_0^{s_3}\frac{du}{2s_2+s_3}e^{-3\Omega(u,s_3)}e^{-2\Omega(s_3,s_2)}\frac{1}{\lambda(t_2)}e^{-\Omega(s_2,t_2)}+
    2\int_{s_3}^{s_2}\frac{du}{2s_2+s_3}e^{-\Omega(u,s_2)}\frac{1}{\lambda(t_2)}e^{-\Omega(s_2,t_2)}\\
\end{align*}
For $t_3=s_3; t_2<s_2$:
    % first summand
\begin{align*}
    &\frac{1}{2s_2+s_3}\frac{1}{\lambda_{\alpha(t_2)}}e^{-\Omega(s_3,t_2)}
    \left\{\sum_{i=0}^{\alpha(s_3)-1}e^{-3\Omega(T_{i+1},s_3)}
        \left[1-e^{-\frac{3\Delta_i}{\lambda_i}}\right]\frac{\lambda_i}{3}+
    \left[1-e^{-\frac{3\left(s_3-T_{\alpha(s_3)}\right)}{\lambda_{\alpha(s_3)}}}\right]
        \frac{\lambda_{\alpha(s_3)}}{3}\right\}+\\
    % second summand
        &\qquad\frac{2}{2s_2+s_3}\frac{1}{\lambda_{\alpha(t_2)}}\\
    &\qquad\Bigg\{e^{-2\Omega(T_{\alpha(s_3)+1},t_2)}\left[1-e^{-\frac{2\left(T_{\alpha(s_3)+1}-s_3\right)}{\lambda_{\alpha(s_3)}}}\right]\frac{\lambda_{\alpha(s_3)}}{2}+\sum_{i=\alpha(s_3)+1}^{\alpha(t_2)-1}e^{-2\Omega\left(T_{i+1},t_2\right)}\left[1-e^{-\frac{2\Delta_i}{\lambda_i}}\right]\frac{\lambda_i}{2}\\
    &\qquad\qquad+\left[1-e^{-\frac{2\left(t_2-T_{\alpha(t_2)}\right)}{\lambda_{\alpha(t_2)}}}\right]\frac{\lambda_{\alpha(t_2)}}{2}
    \Bigg\}\\
    % original
    &\int_0^{s_3}\frac{du}{2s_2+s_3}e^{-3\Omega(u,s_3)}\frac{1}{\lambda(t_2)}e^{-2\Omega(s_3,t_2)}+
    2\int_{s_3}^{t_2}\frac{du}{2s_2+s_3}\frac{1}{\lambda(t_2)}e^{-2\Omega(u,t_2)}\\
    % for...
\end{align*}
For $t_3<s_3; t_2=s_3$:
\begin{align*}
    &\frac{1}{2s_2+s_3}\frac{2}{\lambda_{\alpha(t_3)}}
    \left\{\sum_{i=0}^{\alpha(t_3)-1}e^{-3\Omega(T_{i+1},t_3)}
        \left[1-e^{-\frac{3\Delta_i}{\lambda_i}}\right]\frac{\lambda_i}{3}+
    \left[1-e^{-\frac{3\left(t_3-T_{\alpha(t_3)}\right)}{\lambda_{\alpha(t_3)}}}\right]
        \frac{\lambda_{\alpha(t_3)}}{3}\right\}\\
    % original
    &\int_0^{t_3}\frac{du}{2s_2+s_3}e^{-3\Omega(u,t_3)}\frac{2}{\lambda(t_3)}\\
\end{align*}
For $t_3<s_3; t_2=s_2$:
\begin{align*}
    &\frac{2}{2s_2+s_3}\frac{2}{\lambda_{\alpha(t_3)}}
    \left\{\sum_{i=0}^{\alpha(t_3)-1}e^{-3\Omega(T_{i+1},t_3)}
        \left[1-e^{-\frac{3\Delta_i}{\lambda_i}}\right]\frac{\lambda_i}{3}+
    \left[1-e^{-\frac{3\left(t_3-T_{\alpha(t_3)}\right)}{\lambda_{\alpha(t_3)}}}\right]
        \frac{\lambda_{\alpha(t_3)}}{3}\right\}\\
    % original
    &2\int_0^{t_3}\frac{du}{2s_2+s_3}\frac{2}{\lambda(t_3)}e^{-3\Omega(u,t_3)}\\
    % for...
\end{align*}
For $t_3>s_3; t_2=s_2$:
\begin{align*}
    &\frac{2}{2s_2+s_3}\frac{2}{\lambda_{\alpha(t_3)}}e^{-2\Omega(s_3,t_3)}
    \left\{\sum_{i=0}^{\alpha(s_3)-1}e^{-3\Omega(T_{i+1},s_3)}
        \left[1-e^{-\frac{3\Delta_i}{\lambda_i}}\right]\frac{\lambda_i}{3}+
    \left[1-e^{-\frac{3\left(s_3-T_{\alpha(s_3)}\right)}{\lambda_{\alpha(s_3)}}}\right]
        \frac{\lambda_{\alpha(s_3)}}{3}\right\}\\
    % original
    &2\int_0^{s_3}\frac{du}{2s_2+s_3}e^{-3\Omega(u,s_3)}\frac{2}{\lambda(t_3)}e^{-2\Omega(s_3,t_3)}\\
\end{align*}
For $t_3=s_2; t_2>s_2$:
\begin{align*}
    &\frac{2}{2s_2+s_3}e^{-2\Omega(s_3,s_2)}\frac{1}{\lambda_{\alpha(t_2)}}e^{-\Omega(s_2,t_2)}
    \left\{\sum_{i=0}^{\alpha(s_3)-1}e^{-3\Omega(T_{i+1},s_3)}
        \left[1-e^{-\frac{3\Delta_i}{\lambda_i}}\right]\frac{\lambda_i}{3}+
    \left[1-e^{-\frac{3\left(s_3-T_{\alpha(s_3)}\right)}{\lambda_{\alpha(s_3)}}}\right]
        \frac{\lambda_{\alpha(s_3)}}{3}\right\}\\
    % original
    &2\int_0^{s_3}\frac{du}{2s_2+s_3}e^{-3\Omega(u,s_3)}e^{-2\Omega(s_3,s_2)}\frac{1}{\lambda(t_2)}e^{-\Omega(s_2,t_2)}
\end{align*}
For $t_3=s_3; t_2=s_2$:
\begin{align*}
    &\frac{1}{2s_2+s_3}
    \left\{s_3 - \sum_{i=0}^{\alpha(s_3)-1}e^{-3\Omega(T_{i+1},s_3)}
        \left[1-e^{-\frac{3\Delta_i}{\lambda_i}}\right]\frac{\lambda_i}{3}-
    \left[1-e^{-\frac{3\left(s_3-T_{\alpha(s_3)}\right)}{\lambda_{\alpha(s_3)}}}\right]
        \frac{\lambda_{\alpha(s_3)}}{3}\right\}+\\
    &\frac{1}{2s_2+s_3}\frac{1}{2}\left[1-e^{-2\Omega(s_3,s_2)}\right]
    \left\{\sum_{i=0}^{\alpha(s_3)-1}e^{-3\Omega(T_{i+1},s_3)}
        \left[1-e^{-\frac{3\Delta_i}{\lambda_i}}\right]\frac{\lambda_i}{3}+
    \left[1-e^{-\frac{3\left(s_3-T_{\alpha(s_3)}\right)}{\lambda_{\alpha(s_3)}}}\right]
        \frac{\lambda_{\alpha(s_3)}}{3}\right\}+\\
        &\frac{1}{2s_2+s_3}\Bigg[s_2-s_3-e^{-2\Omega(T_{\alpha(s_3)+1},t_2)}\Bigg(1-e^{-\frac{2\left(T_{\alpha(s_3)+1}-s_3\right)}{\lambda_{\alpha(s_3)}}}\Bigg)\frac{\lambda_{\alpha(s_3)}}{2}-\sum_{i=\alpha(s_3)+1}^{\alpha(t_2)-1}e^{-2\Omega\left(T_{i+1},t_2\right)}\Bigg(1-e^{-\frac{2\Delta_i}{\lambda_i}}\Bigg)\frac{\lambda_i}{2}\\
    &\qquad\qquad-\Bigg(1-e^{-\frac{2\left(t_2-T_{\alpha(t_2)}\right)}{\lambda_{\alpha(t_2)}}}\Bigg)\frac{\lambda_{\alpha(t_2)}}{2}
    \Bigg]\\
    \\
    % original
    &3\int_0^{s_3}\frac{du}{2s_2+s_3}\frac{1}{3}\left[1-e^{-3\Omega(u,s_3)}\right]+\int_0^{s_3}\frac{du}{2s_2+s_3}e^{-3\Omega(u,s_3)}\frac{1}{2}\left[1-e^{-2\Omega(s_3,s_2)}\right]+2\int_{s_3}^{s_2}\frac{du}{2s_2+s_3}\frac{1}{2}\left[1-e^{-2\Omega(u,s_2)}\right].
\end{align*}

\section{Discrete approximation to the triploid SMC' coalescent process}

In order to construct a hidden Markov model (HMM) to infer demography, it is
necessary to discretize the triploid coalescent process described above.

Let the discrete state $(i,j)$, $i<j$, correspond to the continuous states in which ${T_i
< t_3 < T_{i+1}}$ and ${T_j < t_2 < T_{j+1}}$. We first calculate the equilibrium
probability that the coalescent process is in $(i,j)$:

\begin{align}
    \begin{split}
        \pi_{i,j} &=
        \int_{T_i}^{T_{i+1}}\int_{T_j}^{T_{j+1}}\frac{1}{\lambda_i\lambda_j}e^{-3\Omega(0,t_3)}e^{-2\Omega(t_3,t_2)}dt_2dt_3\\
        &= \frac{1}{\lambda_i\lambda_j}\int_{T_i}^{T_{i+1}}e^{-3\Omega(0,t_3)}\int_{T_j}^{T_{j+1}}e^{-2\Omega(t_3,t_2)}dt_2dt_3\\
        &= \frac{1}{\lambda_i\lambda_j}e^{-3\Omega(0,T_i)}\int_{T_i}^{T_{i+1}}e^{-3\Omega(T_i,t_3)}e^{-2\Omega(t_3,T_j)}dt_3
            \int_{T_j}^{T_{j+1}}e^{-2\Omega(T_j,t_2)}dt_2\\
        &= \frac{1}{\lambda_i\lambda_j}e^{-3\Omega(0,T_i)}\int_{T_i}^{T_{i+1}}e^{-3\Omega(T_i,t_3)}e^{-2\Omega(t_3,T_j)}dt_3
            \int_{T_j}^{T_{j+1}}e^{-\frac{2(t_2-T_j)}{\lambda_j}}dt_2\\
        &= \frac{1}{\lambda_i\lambda_j}e^{-3\Omega(0,T_i)}\frac{\lambda_j}{2}\left[1-e^{-\frac{2\Delta_j}{\lambda_j}}\right]
            \int_{T_i}^{T_{i+1}}e^{-3\Omega(T_i,t_3)}e^{-2\Omega(t_3,T_j)}dt_3\\
        &= \frac{1}{\lambda_i\lambda_j}e^{-3\Omega(0,T_i)}\frac{\lambda_j}{2}\left[1-e^{-\frac{2\Delta_j}{\lambda_j}}\right]
            e^{-2\Omega(T_{i+1},T_j)}
            \int_{T_i}^{T_{i+1}}e^{-3\Omega(T_i,t_3)}e^{-2\Omega(t_3,T_{i+1})}dt_3\\
        &= \frac{1}{\lambda_i\lambda_j}e^{-3\Omega(0,T_i)}\frac{\lambda_j}{2}\left[1-e^{-\frac{2\Delta_j}{\lambda_j}}\right]
            e^{-2\Omega(T_{i+1},T_j)}
            \int_{T_i}^{T_{i+1}}e^{-\frac{3(t_3-T_i)}{\lambda_i}}e^{-\frac{2(T_{i+1}-t_3)}{\lambda_i}}dt_3\\
        &= \frac{1}{\lambda_i\lambda_j}e^{-3\Omega(0,T_i)}\frac{\lambda_j}{2}\left[1-e^{-\frac{2\Delta_j}{\lambda_j}}\right]
            e^{-2\Omega(T_{i+1},T_j)}
            \lambda_i \left(e^{\frac{-2\Delta_i}{\lambda_i}}-e^{\frac{-3\Delta_i}{\lambda_i}}\right)\\
        &= \frac{1}{2}e^{-3\Omega(0,T_i)}e^{-2\Omega(T_{i+1},T_j)}
            \left(e^{\frac{-2\Delta_i}{\lambda_i}}-e^{\frac{-3\Delta_i}{\lambda_i}}\right)
            \left[1-e^{-\frac{2\Delta_j}{\lambda_j}}\right].
    \end{split}
    \label{eq:pidiscrete}
\end{align}

Next, we calculate marginal expectations for $t_3$ and $t_2$ given that the
continuous process is in interval represented by $(i,j)$. The marginal
expectation of $t_3$ in the interval $(i,j)$ is 

\begin{align}
    \begin{split}
        \E_{i,j}[t_3] = \E\big[t_3|&t_3 \in [T_i T_{i+1}),t_2 \in [T_j,T_{j+1}) \big]\\
        &= \frac{1}{\pi_{i,j}}\int_{T_i}^{T_{i+1}}\int_{T_j}^{T_{j+1}}t_3\pi(t_3,t_2)dt_2dt_3\\
        &= \frac{1}{\pi_{i,j}}\int_{T_i}^{T_{i+1}}\int_{T_j}^{T_{j+1}}
            \frac{t_3}{\lambda_i \lambda_j}e^{-3\Omega(0,t_3)}e^{-2\Omega(t_3,t_2)}dt_2dt_3\\
        &=\frac{1}{\pi_{i,j}\lambda_i\lambda_j}e^{-3\Omega(0,T_i)}
            \int_{T_i}^{T_{i+1}}t_3e^{-3\Omega(T_i,t_3)}\int_{T_j}^{T_{j+1}}e^{-2\Omega(t_3,t_2)}dt_2dt_3\\
        &=\frac{1}{\pi_{i,j}\lambda_i\lambda_j}e^{-3\Omega(0,T_i)}
            \int_{T_i}^{T_{i+1}}t_3e^{-3\Omega(T_i,t_3)}e^{-2\Omega(t_3,T_j)}dt_3\int_{T_j}^{T_{j+1}}e^{-2\Omega(T_j,t_2)}dt_2\\
        &=\frac{1}{\pi_{i,j}\lambda_i\lambda_j}e^{-3\Omega(0,T_i)}
            \int_{T_i}^{T_{i+1}}t_3e^{-3\Omega(T_i,t_3)}e^{-2\Omega(t_3,T_j)}dt_3\int_{T_j}^{T_{j+1}}e^{-\frac{2(t_2-T_j)}{\lambda_j}}dt_2\\
        &=\frac{1}{\pi_{i,j}\lambda_i\lambda_j}e^{-3\Omega(0,T_i)}\frac{\lambda_j}{2}\left(1-e^{-\frac{2\Delta_j}{\lambda_j}}\right)
            \int_{T_i}^{T_{i+1}}t_3e^{-3\Omega(T_i,t_3)}e^{-2\Omega(t_3,T_j)}dt_3\\
        &=\frac{1}{\pi_{i,j}\lambda_i\lambda_j}e^{-3\Omega(0,T_i)}
            \frac{\lambda_j}{2}\left(1-e^{-\frac{2\Delta_j}{\lambda_j}}\right)e^{-2\Omega(T_{i+1},T_j)}
            \int_{T_i}^{T_{i+1}}t_3e^{-3\Omega(T_i,t_3)}e^{-2\Omega(t_3,T_{i+1})}dt_3\\
        &=\frac{1}{\pi_{i,j}\lambda_i\lambda_j}e^{-3\Omega(0,T_i)}
            \frac{\lambda_j}{2}\left(1-e^{-\frac{2\Delta_j}{\lambda_j}}\right)e^{-2\Omega(T_{i+1},T_j)}
            \lambda_i\left[(\lambda_i+T_i)e^{\frac{-2\Delta_i}{\lambda_i}}-(\lambda_i+T_{i+1})e^{\frac{-3\Delta_i}{\lambda_i}}\right]\\
        &=\frac{1}{2\pi_{i,j}}e^{-3\Omega(0,T_i)}
            \left(1-e^{-\frac{2\Delta_j}{\lambda_j}}\right)e^{-2\Omega(T_{i+1},T_j)}
            \left[(\lambda_i+T_i)e^{\frac{-2\Delta_i}{\lambda_i}}-(\lambda_i+T_{i+1})e^{\frac{-3\Delta_i}{\lambda_i}}\right].\\
    \end{split}
    \label{eq:Et3}
\end{align}
The marginal expectation of $t_2$ in $(i,j)$ is 
\begin{align}
    \begin{split}
        \E_{i,j}[t_2] = \E\big[t_2|&t_3 \in [T_i T_{i+1}),t_2 \in [T_j,T_{j+1}) \big]\\
        &= \frac{1}{\pi_{i,j}}\int_{T_i}^{T_{i+1}}\int_{T_j}^{T_{j+1}}t_2\pi(t_3,t_2)dt_2dt_3\\
        &= \frac{1}{\pi_{i,j}}\int_{T_i}^{T_{i+1}}\int_{T_j}^{T_{j+1}}
            \frac{t_2}{\lambda_i \lambda_j}e^{-3\Omega(0,t_3)}e^{-2\Omega(t_3,t_2)}dt_2dt_3\\
        &= \frac{1}{\pi_{i,j}\lambda_i\lambda_j}e^{-3\Omega(0,T_i)}e^{-2\Omega(T_{i+1},T_j)}
            \int_{T_i}^{T_{i+1}}e^{-3\Omega(T_i,t_3)}e^{-2\Omega(t_3,T_{i+1})}dt_3
            \int_{T_j}^{T_{j+1}}t_2e^{-2\Omega(T_j,t_2)}dt_2\\
        &= \frac{1}{\pi_{i,j}\lambda_i\lambda_j}e^{-3\Omega(0,T_i)}e^{-2\Omega(T_{i+1},T_j)}
            \int_{T_i}^{T_{i+1}}e^{-\frac{3(t_3-T_i)}{\lambda_i}}e^{-\frac{2(T_{i+1}-t_3)}{\lambda_i}}dt_3
            \int_{T_j}^{T_{j+1}}t_2e^{-2\Omega(T_j,t_2)}dt_2\\
        &= \frac{1}{\pi_{i,j}\lambda_i\lambda_j}e^{-3\Omega(0,T_i)}e^{-2\Omega(T_{i+1},T_j)}
            \lambda_i \left(e^{\frac{-2\Delta_i}{\lambda_i}}-e^{\frac{-3\Delta_i}{\lambda_i}}\right)
            \int_{T_j}^{T_{j+1}}t_2e^{-2\Omega(T_j,t_2)}dt_2\\
        &= \frac{1}{\pi_{i,j}\lambda_i\lambda_j}e^{-3\Omega(0,T_i)}e^{-2\Omega(T_{i+1},T_j)}
            \lambda_i \left(e^{\frac{-2\Delta_i}{\lambda_i}}-e^{\frac{-3\Delta_i}{\lambda_i}}\right)
            \int_{T_j}^{T_{j+1}}t_2e^{-\frac{2(t_2-T_j)}{\lambda_j}}dt_2\\
        &= \frac{1}{\pi_{i,j}\lambda_i\lambda_j}e^{-3\Omega(0,T_i)}e^{-2\Omega(T_{i+1},T_j)}
            \lambda_i \left(e^{\frac{-2\Delta_i}{\lambda_i}}-e^{\frac{-3\Delta_i}{\lambda_i}}\right)
            \frac{\lambda_j}{4}\left(\lambda_j+2T_j-(\lambda_j +2 T_{j+1}) e^{-\frac{2\Delta_j}{\lambda_j}}\right)\\
        &= \frac{1}{4\pi_{i,j}}e^{-3\Omega(0,T_i)}e^{-2\Omega(T_{i+1},T_j)}
            \left(e^{\frac{-2\Delta_i}{\lambda_i}}-e^{\frac{-3\Delta_i}{\lambda_i}}\right)
            \left(\lambda_j+2T_j-(\lambda_j +2 T_{j+1}) e^{-\frac{2\Delta_j}{\lambda_j}}\right).
    \end{split}
    \label{eq:Et2}
\end{align}

To calculate the discrete-process transition probabilities from $(i,j)$, to
$(k,l)$, we integrate the transition kernel \eqref{eq:piecewisenocoalintegral}
over the interval corresponding to $(k,l)$, replacing $s_3$ and $s_2$ with
their conditional expectations $\E_{i,j}[s_3]$ and $\E_{i,j}[s_2]$
respectively. Thus

\begin{align}
    q\Big((k,l)\,|\,(i,j)\Big) &= 
    \int_{T_k}^{T_{k+1}}\int_{T_l}^{T_{l+1}}q\Big((t_3,t_2)|\left(\E_{i,j}[s_3],\E_{i,j}[s_2]\right)\Big)dt_2dt_3\\
    &= 
\end{align}

\end{document}
